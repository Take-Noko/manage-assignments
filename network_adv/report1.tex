\documentclass{jlreq}
\usepackage{listings,jvlisting}

\title{ネットワーク設計/応用 \\ \vspace{0.3cm} 演習1 レポート}
\author{学籍番号1270381 宮本武}


\begin{document}
\maketitle

\section{演習を通して学んだこと}
\subsection{演習内容}
今回の演習では,サイバーセキュリティネクサス(CYNEX)のネットワーク内のサーバに
工科大の個人端末からリモート接続を行うことで,
OSコマンドインジェクションによる攻撃及び防御の手法を学んだ.

具体的な演習内容としては,
\begin{description}
    \item[攻撃手法]
    CYNEXのネットワーク内にある講師用コンテンツサーバ(ドメイン検索サイト)に対して,
    同じネットワークに属すコンピュータからサイトにアクセスし,
    OSコマンドインジェクションを実行することで,
    サーバ内のユーザ名,パスワードといった機密情報を不正に取得する.

    \vspace{0.3cm}
    \item[防御手法] 
    OSコマンドインジェクションが実行される原因を排除し,
    健全なサーバ運営が行える様にプログラムを改良する,
\end{description}
であった.

\subsection{インジェクション実行結果}
講師用コンテンツサーバのドメイン検索サイトにて,
ドメイン名を入力すべきフォームに
\begin{lstlisting}[frame={tblr}]
; cat <ファイルパス>
\end{lstlisting}
といった形式で機密情報が入ったファイルのパスを指定してフォームを送信したところ,
サーバからの返信欄にドメイン検索結果に続いて,機密情報が入ったファイルの内容が表示された.
\subsection{原因と危険性}
インジェクションの実行結果に機密情報が入ってしまっていることから,
情報漏洩が発生している.
この原因としては

\subsection{対処方法}


\section{感想}
\end{document}